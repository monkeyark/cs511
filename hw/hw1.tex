\documentclass[12pt]{article}


\usepackage{fullpage}
\usepackage{graphicx}
\usepackage{amsmath,amsthm,amssymb,amsfonts,epic,epsfig,latexsym,enumerate}
\usepackage{enumitem}
\usepackage[titlenotnumbered,linesnumbered,noend,plain]{algorithm2e} 
\usepackage{listings}
\usepackage{fullpage}
\usepackage{array}
\usepackage{mathtools}
\usepackage{xcolor}
\newtheorem{lemma}{Lemma}
\usepackage{url}
\SetKwProg{Fn}{}{}{}
\DeclarePairedDelimiter{\ceil}{\lceil}{\rceil}
\DeclarePairedDelimiter{\floor}{\lfloor}{\rfloor}
\date{}

\begin{document}

\begin{center}
  \textbf{COMS 5110: Assignment 1}\\
  \textbf{Due: Feb $10^{th}$, 11:59pm}\\
  \textbf{Total Points: 50}
\end{center}


\paragraph{Late submission policy.\ }
Any assignment submission that is late by not more than two business days from the deadline will be accepted with 20\% penalty for each business day. That is, if a homework is due on Friday at 11:59 PM, then a Monday submission gets 20\% penalty and a Tuesday submission gets another 20\% penalty.  After Tuesday no late submissions are accepted.

\paragraph{Submission format.\ }
Homework solutions will have to be typed. You can use word, LaTeX, or any other type-setting tool to type your solution. Your submission file should be in pdf format. Do \textbf{NOT} submit a photocopy of handwritten homework except for diagrams that can be hand-drawn and scanned. We reserve the right \textbf{NOT} to grade homework that does not follow the formatting requirements.
Name your submission
file: \texttt{<Your-net-id>-5110-hw1.pdf}. For instance, if your netid
is \texttt{asterix}, then your submission file will be named
\texttt{asterix-5110-hw1.pdf}.
Each student must hand in their own assignment. If you discussed the homework or solutions with others, a list of collaborators must be included with each submission. Each of the collaborators has to write the solutions in their own words (copies are not allowed).

\paragraph{General Requirements}
\begin{itemize}
    \item When proofs are required, do your best to make them both clear and rigorous.
    \item Even when proofs are not required, you should justify your answers and explain your work.
   % \item When asked to present a construction, you should show the correctness of the construction.
\end{itemize}

\paragraph{Some Useful (in)equalities}
\begin{itemize}
    \item $\sum_{i=1}^n i= \frac{n(n+1)}{2}$
    \item $\sum_{i=1}^ni^2 = \frac{n(n+1)(2n+1)}{6}$
    \item $2^{\log_2n}=n$, $a^{\log_bn}=n^{\log_ba}$, $n^{n/2}\leq n! \leq n^n$, $\log x^a = a\log x$
    \item $\log (a\times b) = \log a + \log b$, $\log(a/b) = \log a - \log b$
    \item $a + ar + ar^2 + ... + ar^{n-1} = \frac{a(r^n-1)}{r-1}$
    \item $1 + \frac{1}{2} + \frac{1}{2^2} + ... + \frac{1}{2^n} = 2(1-\frac{1}{2^{n+1}})$
    \item $1 + 2 + 4 + ... + 2^n = 2^{n+1}-1$
\end{itemize}

\hrule

\subsection*{Problem 1} (10 pints) {\bf Model of Computation} 
Consider the following algorithm (written in pseudocode):
\smallskip

\begin{algorithm}[H]
\Fn(){Alg(A, n)}{
\KwIn{Array A[1 .. n] of positive integers each taking 50 bits of memory}
\SetAlgoLined
\SetNoFillComment
\DontPrintSemicolon
$min = 0$ \\
$prod = 1$ \\
\For{$i = 1$ to $n$}{
$prod = 2 * prod$ \\
\eIf{$prod > A[i]$}{
$min = A[i]$ \\
}{
$min = prod$ \\
}
}
}
\end{algorithm}



\smallskip
In the {\em Random-Access Machine (RAM)} model, each word has $c \log_2 n$ bits for some constant $c \geq 1$ so that each word can hold the value of $n$. Determine if the results stored in $prod$ and $min$ in the above algorithm each fit into a word in the RAM model. You need to justify your answer by following the definition of the big O notation in your proof. Let $p(n)$ and $m(n)$ be the results saved in $prod$ and $min$, respectively, at the end of the algorithm. The function $p(n)$ fits into a word in the RAM model if $p(n)$ is in $O(\log_2 n)$; $p(n)$ does not fit into a word in the RAM model if $p(n)$ is not in $O(\log_2 n)$.

\subsection*{Problem 2} (10 pints) {\bf Mathematical Induction} 
Show by induction that $8 \log_2 n \leq n$ for all $n \geq n_0$ for some $n_0  > 0$.

\subsection*{Problem 3} (10 pints) {\bf Sizes of Left and Right Subarrays in Merge Sort}
Let $A[p .. r]$ be a subarray of $A[1 .. n]$ with $1 \leq p < r \leq n$. In the recursive algorithm MERGE-SORT$(A, p , r)$, the subarray $A[p .. r]$ is partitioned into left and right subarrays $A[p .. q]$ and $A[q+1 .. r]$ with $q = \floor*{(p+r)/2}$. Let $m = r - p + 1$ be the size of the subarray $A[p .. r]$. Show that the size of $A[p .. q]$ is $q- p + 1 = \ceil*{\frac{m}{2}}$, and that the size of $A[q+1 .. r]$ is $r - (q+1) + 1 = \floor*{\frac{m}{2}}$. You just need to prove it for the case where $p+r$ is odd, that is, $p + r = 2 y + 1$ for some integer $y$.

\subsection*{Problem 4} (20 points) {\bf Solving recurrence by induction} 
Use mathematical induction to show that $T(n) \leq c n - b \log_2 n - a$ for all $n \geq n_0$ for some constants $n_0 > 0$, $c > 0$, $b > 0$ and $a > 0$, where $T(n)$ is defined in the following recurrence:

$T(1) = d$,

$T(n) = T( \ceil*{\frac{n}{2}} ) + T( \floor*{\frac{n}{2}} ) + d \log_2 n$ if $n > 1$, where $d$ is a positive constant.

\end{document}
